\documentclass{article}
\usepackage{graphicx}
\usepackage{outlines}
\usepackage{helvet}
\usepackage{microtype}

\renewcommand{\familydefault}{\sfdefault}
\title{BIOL BC3300: Pipeline Assignment}
\date{Due: April 7, 2023 at 11:59 PM}

\begin{document}

\maketitle

\section*{Bioinformatic pipelines (40 pts)}

\subsection*{Assignment}
Read the \texttt{Kondratyeva\_2020} and \texttt{David\_2019} papers. Submit typed answers to the paper-specific questions on Courseworks.\\

The papers can be found in \textbf{Files $\to$ Papers $\to$ Examples}

\section*{Paper 1 (20 pts): \texttt{David\_2019.pdf}}
\begin{itemize}
    \item \textbf{(3 pt)} What is the overall research question? You should be able to state it concisely in a single sentence.
    \item \textbf{(3 pt)} Describe the entire dataset that was used in this study, where it came from, and where it can be found now (if different than where it came from). If applicable, be specific with databases and accession IDs.
    \item \textbf{(4 pt)} Describe the full bioinformatic pipeline used to generate the data presented in Figures 2a and 2b, beginning with the sequencing platform. For each step, include the specific tools or program used, what they were used for, and any criteria that was implemented to filter out data along the way. How is Fig 2b different from 2a? 
    \item \textbf{(3 pt)} How do you interpret Fig 5a? What additional data is used here compared to Figure 2?
    \item \textbf{(3 pt)} What do the authors mean by a `major clonal lineage'?
    \item \textbf{(4 pt)} What are the main conclusions of this study? Do you agree? Why or why not?
\end{itemize}

\section*{Paper 2 (20 pts): \texttt{Kondrayeva\_2020.pdf}}
\begin{itemize}
    \item \textbf{(3 pt)} What is the overall research question? You should be able to state it concisely in a single sentence.
    \item \textbf{(3 pt)} Describe the entire dataset that was used in this study, where it came from, and where it can be found now (if different than where it came from). If applicable, be specific with databases and accession IDs.
    \item \textbf{(4 pt)} Describe the full bioinformatic pipeline used to generate the data presented in Figure 1, beginning with the sequencing platform. For each step, include the specific tools or program used, what they were used for, and any criteria that was implemented to filter out data along the way.
    \item \textbf{(3 pt)} What is the purpose of the shaded purple regions in Figure 4a (and b), and how does presenting the data in this way contribute to the main results in this figure?
    \item \textbf{(3 pt)} What data is used in Figure 5, and what is the main takeaway from this tree?
    \item \textbf{(4 pt)} What are the main conclusions of this study? Do you agree? Why or why not?
\end{itemize}

\end{document}
