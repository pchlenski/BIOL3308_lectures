\documentclass{beamer}
\usetheme{chlenski}
\usepackage[utf8]{inputenc}
\usepackage{graphicx}
\usepackage{outlines}
\usepackage{hyperref}
\usepackage[nodisplayskipstretch]{setspace}

\title{Intro to Microbial Genomics}
\subtitle{BIOL BC3308, Lecture 4}
\author{Philippe Chlenski}
\date{February 7, 2023}

\renewcommand{\c}[1]{\begin{center}#1\end{center}}
\newcommand{\blu}[1]{\textcolor{blue}{\textbf{#1}}}
\newcommand{\red}[1]{\textcolor{red}{\textbf{#1}}}
\newcommand{\yel}[1]{\textcolor{yellow}{\textbf{#1}}}
\newcommand{\grn}[1]{\textcolor{dark-green}{\textbf{#1}}}
\newcommand{\prp}[1]{\textcolor{purple}{\textbf{#1}}}
\newcommand{\gr}[2][.95]{\c{\includegraphics[width=#1\textwidth]{#2}}}

% \setbeameroption{show notes}

\begin{document}

\begin{frame}[plain]
\titlepage
\end{frame}

\begin{frame}{Outline}
\tableofcontents
\end{frame}

\section{Background and motivation}

\begin{frame}{Overview}
\begin{outline}
\1[] Questions:
    \2 What is genome assembly?
\1[] Theory:
    \2 Steps in genome assembly
    \2 De Brujin graphs
\1[] Practical considerations:
    \2 Trimming sequences
    \2 Determining the quality
    \2 Assembling genomes
\end{outline}
\end{frame}

\begin{frame}{What is genome assembly?}
\gr{l4_figs/s3_intro.png}
\end{frame}

\begin{frame}{Sequencing by synthesis video}
\url{https://www.youtube.com/watch?v=fCd6B5HRaZ8&t=3s}
\end{frame}

\begin{frame}{Illumina short reads and FASTQ files}
\begin{outline}
    \1[] High-throughput sequencing reads are usually output from sequencing facilities as text files in a format called “FASTQ” or “fastq”. 
    \1[] This format is an extension of the commonly used FASTA format
    \1[] Designed to handle base quality metrics output from sequencing machines
    \1[] Uses 4 lines for each sequence, and these 4 lines are stacked on top of each other in text files output by sequencing workflows.
\end{outline}
\gr{l4_figs/s5_read.png}
\end{frame}

\begin{frame}{Basic steps for assembling a genome}
\begin{enumerate}
    \item Basic DNA sequence cleanup and evaluation
    \item Contig building
    \item Scaffolding
    \item Post-assembly processing and analysis
\end{enumerate}
\end{frame}

\begin{frame}{1. Basic DNA sequence cleanup and evaluation}
What are we looking for?
\begin{outline}
    \1 Is the DNA sequence high quality? Does it need to be trimmed?
    \1 Evaluate libraries for coverage
    \1 Any additional sequence preparation steps
\end{outline}
\end{frame}

\begin{frame}{1. Basic DNA sequence cleanup and evaluation, cont}
\gr{l4_figs/s8_scores.png}
$Q = -10 \log_{10}(P)$, where $P$ is the probability that a base call is erroneous
\end{frame}

\begin{frame}{1. Basic DNA sequence cleanup and evaluation, cont}
\gr{l4_figs/s9_fastq.png}
Cleanup: trim reads for bases that fall below quality score
\end{frame}

\b

\end{document}