\documentclass{beamer}
\usetheme{chlenski}
\usepackage[utf8]{inputenc}
\usepackage{graphicx}
\usepackage{outlines}
\usepackage{hyperref}
\usepackage[nodisplayskipstretch]{setspace}

\title{Intro to Microbial Genomics}
\subtitle{BIOL BC3308, Lecture 4}
\author{Philippe Chlenski}
\date{February 7, 2023}

\renewcommand{\c}[1]{\begin{center}#1\end{center}}
\newcommand{\blu}[1]{\textcolor{blue}{\textbf{#1}}}
\newcommand{\red}[1]{\textcolor{red}{\textbf{#1}}}
\newcommand{\yel}[1]{\textcolor{yellow}{\textbf{#1}}}
\newcommand{\grn}[1]{\textcolor{dark-green}{\textbf{#1}}}
\newcommand{\prp}[1]{\textcolor{purple}{\textbf{#1}}}
\newcommand{\gr}[2][.95]{\c{\includegraphics[width=#1\textwidth]{#2}}}

% \setbeameroption{show notes}

\begin{document}

\begin{frame}[plain]
\titlepage
\end{frame}

\begin{frame}{Outline}
\tableofcontents
\end{frame}

\section{Background and motivation}

\begin{frame}{Overview}
\begin{outline}
\1[] Questions:
    \2 What is genome assembly?
\1[] Theory:
    \2 Steps in genome assembly
    \2 De Brujin graphs
\1[] Practical considerations:
    \2 Trimming sequences
    \2 Determining the quality
    \2 Assembling genomes
\end{outline}
\end{frame}

\begin{frame}{What is genome assembly?}
\gr{l4_figs/s3_intro.png}
\end{frame}

\begin{frame}{Sequencing by synthesis video}
\url{https://www.youtube.com/watch?v=fCd6B5HRaZ8&t=3s}
\end{frame}

\begin{frame}{Illumina short reads and FASTQ files}
\begin{outline}
    \1[] High-throughput sequencing reads are usually output from sequencing facilities as text files in a format called “FASTQ” or “fastq”. 
    \1[] This format is an extension of the commonly used FASTA format
    \1[] Designed to handle base quality metrics output from sequencing machines
    \1[] Uses 4 lines for each sequence, and these 4 lines are stacked on top of each other in text files output by sequencing workflows.
\end{outline}
\gr{l4_figs/s5_read.png}
\end{frame}

\begin{frame}{Basic steps for assembling a genome}
\begin{enumerate}
    \item Basic DNA sequence cleanup and evaluation
    \item Contig building
    \item Scaffolding
    \item Post-assembly processing and analysis
\end{enumerate}
\end{frame}

\section{Sequence cleanup}

\begin{frame}{What are we looking for?}
\begin{outline}
    \1 Is the DNA sequence high quality? Does it need to be trimmed?
    \1 Evaluate libraries for coverage
    \1 Any additional sequence preparation steps
\end{outline}
\end{frame}

\begin{frame}{Phred quality scores}
\gr{l4_figs/s8_scores.png}
$Q = -10 \log_{10}(P)$, where $P$ is the probability that a base call is erroneous
\end{frame}

\begin{frame}{FASTQC for quality control}
\gr{l4_figs/s9_fastqc.png}
Cleanup: trim reads for bases that fall below quality score
\end{frame}

\begin{frame}{Genome coverage}
\gr[0.5]{l4_figs/s10_coverage.png}
\end{frame}

\begin{frame}{Genome coverage, cont}
\begin{columns}
\begin{column}{0.5\textwidth}
    \gr{l4_figs/s11_sequencers.png}
\end{column}
\begin{column}{0.5\textwidth}
    \footnotesize
    \[
        \text{Avg coverage} = \frac{\text{read length} \times \text{read count}}{\text{total genome size}}
    \]
    \blu{Example:} I sequence a 4.6Mb-long genome on one flow cell lane using a miSeq instrument. I use 150bp paired-end reads. If one flow cell lane generates 25 million reads, what is my expected coverage?
    \begin{align*}
        \text{Read length} &= 150 * 2 = 300\\
        \text{Read count} &= 25,000,000\\
        \text{Genome size} &= 4,600,000\\
        \implies \text{Avg. coverage} &= 1630X
    \end{align*}
\end{column}
\end{columns}
\end{frame}

\section{Contig building}

\begin{frame}{What are contigs?}
\blu{Contig:} A \textit{contig}uous, unambiguous stretch of assembled DNA sequence
\begin{columns}
\begin{column}{0.4\textwidth}
    The ends of a contig correspond to:\\
    \bigskip
    \begin{outline}
        \1 Real ends (for linear DNA molecules)
        \1 Dead ends (missing sequence)
        \1 Forks in the road
    \end{outline}
    Can be stored in FASTA or multi-FASTA format
\end{column}
\begin{column}{0.4\textwidth}
    \gr{l4_figs/s12_contig.png}
\end{column}
\end{columns}
\end{frame}

\begin{frame}{Algorithm overview}
\begin{outline}
    \1[] Find all overlaps between reads
    \1[] Build a graph
    \1[] Simplify the graph
    \1[] Traverse the graph
    \2 A graph is a tuple $\{V, E\}$ where $V$ is a set of \red{vertices} and $E$ is a set of \blu{edges}
\end{outline}
\gr{l4_figs/s13_contig_building.png}
\end{frame}

\begin{frame}{De Bruijn graph}
    \blu{De Bruijn graph} assemblers are developed to deal with high-throughput, highly accurate short-read data
    \begin{columns}
    \begin{column}{0.5\textwidth}
        \gr{l4_figs/s0_debruijn.png}
        \tiny Source: \url{https://en.wikipedia.org/wiki/De_Bruijn_graph}
    \end{column}
    \begin{column}{0.5\textwidth}
        \begin{outline}
            \1 Nodes: all possible strings, e.g. all $4^k$ possible $k$-mers
            \1 Edges: if it's possible for the next string to be this node:
                \2 Intuition: take off first character, add any character to the end
                \2 Can't go 000 $\to$ 111 directly. Why?
        \end{outline}
    \end{column}
    \end{columns}
\end{frame}

\begin{frame}{Properties of de Bruijn graphs}
For an idealized de Bruijn graph:
    \begin{outline}
        \1 If $m$ is the size of your alphabet, each vertex has exactly $m$ incoming edges and $m$ outgoing edges
        \1 De Bruijn graphs are \blu{Eulerian}: you can visit each \blu{edge} once and end up where you started
        \1 They are also \red{Hamiltonian}: you can visit each \red{node} once and end up where you started
    \end{outline}
    This allows us to do an \blu{Euler tour}, visiting each node in the graph once. The order of visits tells us the original sequence.
\end{frame}

\begin{frame}{Practical considerations}
When we construct de Bruijn graphs from actual sequencing reads:
\begin{outline}
    \1 We restrict ourselves to a subset of nodes
    \1 We may have multiple edges (use a directed \textit{multi}graph)
\end{outline}
To do a true Euler tour, we need:
\begin{outline}
    \1 Every node to have equal in-degree and out-degree
        \2 For an Euler \textit{trail} (start and end points are different), we allow one pair of nodes $(i, j)$ to have $(in\_degree(i)  - out\_degree(i)) = 1$ and $(out\_degree(j) - in\_degree(j)) = 1$
    \1 All nodes in one strongly connected component
\end{outline}
We use approximate algorithms to extract contigs from messy empirical graphs.
\end{frame}

\begin{frame}{Shredded book reconstruction}
Suppose you accidentally shredded a manuscript of \textit{A Tale of Two Cities}\\
\bigskip
How can we reconstruct it? Suppose we have 5 copies $\times$ 138,656 words / 5 words per fragment = 138,656 fragments drawn at random.\\
\bigskip
\red{Example sentence:} \blu{It was the best of} times, \blu{it was the worst of} times, \grn{it was the age of} wisdom, \grn{it was the age of} foolishness\ldots
\end{frame}

\begin{frame}{Overlaps between fragments are implicitly computed}
\gr{l4_figs/s17_twocities.png}
\end{frame}

\begin{frame}{Completed de Bruijn graph, 5-mers}
\gr{l4_figs/s18_graph.png}
\end{frame}

\begin{frame}{Simplified de Bruijn graph}
\gr{l4_figs/s19_subgraph.png}
\end{frame}

\begin{frame}{Larger graph}
\gr{l4_figs/s20_table.png}
\end{frame}

\begin{frame}{DNA version}
\gr{l4_figs/s21_kmers.png}
\end{frame}

\begin{frame}{De Bruijn graph recap}
\begin{enumerate}
    \item Choose a value for $k$ (or use the $k$ you're given)
    \begin{itemize}
        \item This must be less than your fragment size. For sequencing, fragment size = read length
    \end{itemize}
    \item Get all $k$-mers of your sequence
    \item Set $k$-mers as your nodes, and connect them according to overlaps in $(k-1)$-mers
    \begin{itemize}
        \item Intuitively: if you added a character to the end of your $k$-mer and forgot the first character, where could you possibly end up?
    \end{itemize}
    \item Simplify!
    \begin{itemize}
        \item (Use a computer for this)
    \end{itemize}
\end{enumerate}

See \url{https://homolog.us/Tutorials/book4/p2.1.html} for a worked example
\end{frame}

\begin{frame}{In reality, the graphs can get messy\ldots}
\gr{l4_figs/s23_real_graph.png}OK
\end{frame}

\section{Scaffolding and assessment}

\begin{frame}{Scaffolding intro}
Now you have a huge pile of contigs but you want to make them larger. How? Add context!\\
\bigskip Use other information you might have (e.g. mate pairs) and link together contigs
\gr{l4_figs/s24_scaffolding.png}
\end{frame}

\begin{frame}{Long reads}
\blu{Long read sequencing} technology can help scaffold together contigs from short-read but high-fidelity platforms: 
\gr{l4_figs/s25_long_reads.png}
\end{frame}

\begin{frame}{Assessing your assembly}
\begin{outline}
    \1[] How much total sequence is in the assembly relative to estimated genome size?
    \1[] How many pieces (contigs), and what is the distribution of their sizes?
    \1[] Are the contigs assembled correctly?
    \1[] Are the scaffolds connected in the right order and orientation?
    \1[] Are all the genes I expect to be in the assembly there? Do they appear in the right amounts?
\end{outline}
\end{frame}

\begin{frame}{Assessing your assembly, cont}
\begin{columns}
\begin{column}{0.5\textwidth}
 \begin{outline}
     \1[] \blu{N50:} The most common measure of assembly quality
        \2[] Def: medium contig length
    \1[] Other common metrics:
        \2 Mappable reads
        \2 Total length
        \2 Number of contigs
 \end{outline}
\end{column}
\begin{column}{0.5\textwidth}
    \gr{l4_figs/s27_quality.png}
\end{column}
\end{columns}
\end{frame}

\begin{frame}{Other considerations}
\begin{enumerate}
    \item How large and complex is the genome?
    \item Is my coverage sufficient?
    \item What technology was used to sequence? (Long vs. short reads)
    \item Is there a good reference sequence?
    \item What about repeats?\footnote{It is impossible to resolve repeats that are longer than your read length. Why?}
\end{enumerate}
\end{frame}

\begin{frame}{Lab 4}
Topic: \blu{Bash, genome assembly, and comparison}\\
\bigskip
Work in groups of 2.\\
\bigskip
The lab is to be handed in with the HW at the start of next week.
\end{frame}

\end{document}