\documentclass{beamer}
\usetheme{chlenski}
\usepackage[utf8]{inputenc}
\usepackage{graphicx}
\usepackage{outlines}
\usepackage{hyperref}
\usepackage{babel}
\usepackage[nodisplayskipstretch]{setspace}

\title{Intro to Microbial Genomics}
\subtitle{BIOL BC3308, Lecture 6}
\author{Philippe Chlenski}
\date{February 21, 2023}

\renewcommand{\c}[1]{\begin{center}#1\end{center}}
\newcommand{\blu}[1]{\textcolor{blue}{\textbf{#1}}}
\newcommand{\red}[1]{\textcolor{red}{\textbf{#1}}}
\newcommand{\yel}[1]{\textcolor{yellow}{\textbf{#1}}}
\newcommand{\grn}[1]{\textcolor{dark-green}{\textbf{#1}}}
\newcommand{\prp}[1]{\textcolor{purple}{\textbf{#1}}}
\newcommand{\gr}[2][.95]{\c{\includegraphics[width=#1\textwidth]{#2}}}

% \setbeameroption{show notes}

\begin{document}

\begin{frame}[plain]
\titlepage
\end{frame}

\begin{frame}{Outline}
\tableofcontents
\end{frame}

\section{Background and motivation}

\begin{frame}{Overview}
    \begin{outline}
        \1[] Questions:
            \2 Why are we interested in comparing genomes to a reference?
            \2 How do we map reads efficiently?
        \1[] Theory:
            \2 Suffix arrays
            \2 Burrows Wheeler Transformation
            \2 Variant calling
        \1[] Practice:
            \2 BWA alignments
            \2 SNIPPY mutation detection
    \end{outline}
\end{frame}

\begin{frame}{Personal genomics}
    \ldots
\end{frame}

\section{}

\end{document}