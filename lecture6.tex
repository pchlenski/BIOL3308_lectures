\documentclass{beamer}
\usetheme{chlenski}
\usepackage[utf8]{inputenc}
\usepackage{graphicx}
\usepackage{outlines}
\usepackage{hyperref}
\usepackage{babel}
\usepackage{verbatim}
\usepackage[nodisplayskipstretch]{setspace}

\title{Intro to Microbial Genomics}
\subtitle{BIOL BC3308, Lecture 6}
\author{Philippe Chlenski}
\date{February 21, 2023}

\renewcommand{\c}[1]{\begin{center}#1\end{center}}
\newcommand{\blu}[1]{\textcolor{blue}{\textbf{#1}}}
\newcommand{\red}[1]{\textcolor{red}{\textbf{#1}}}
\newcommand{\yel}[1]{\textcolor{yellow}{\textbf{#1}}}
\newcommand{\grn}[1]{\textcolor{dark-green}{\textbf{#1}}}
\newcommand{\prp}[1]{\textcolor{purple}{\textbf{#1}}}
\newcommand{\gr}[2][.95]{\c{\includegraphics[width=#1\textwidth]{#2}}}

% \setbeameroption{show notes}

\begin{document}

\begin{frame}[plain]
\titlepage
\end{frame}

\begin{frame}{Outline}
\tableofcontents
\end{frame}

\section{Background and motivation}

\begin{frame}{Overview}
    \begin{outline}
        \1[] Questions:
            \2 Why are we interested in comparing genomes to a reference?
            \2 How do we map reads efficiently?
        \1[] Theory:
            \2 Suffix arrays
            \2 Burrows Wheeler Transformation
            \2 Variant calling
        \1[] Practice:
            \2 BWA alignments
            \2 SNIPPY mutation detection
    \end{outline}
\end{frame}

\begin{frame}{Personal genomics}
    \begin{columns}
        \begin{column}{0.5\textwidth}
            Suppose we have a reference sequence and reads, with mutations, from different patients.\\
            \bigskip We want to see whether they have mutations that match certain phenotypes.
        \end{column}
        \begin{column}{0.5\textwidth}
            \gr{l6_figs/s3_snps.png}
            \gr[0.6]{l6_figs/s3_phenotypes.png}
        \end{column}
    \end{columns}
\end{frame}

\begin{frame}{Three strategies for read mapping}
    \begin{enumerate}
        \item Brute force approach
        \item Binary search in suffix arrays
        \item Burrows-Wheeler transform
    \end{enumerate}
\end{frame}

\begin{frame}{Strategy 1: brute force}
    \begin{columns}
        \begin{column}{0.5\textwidth}
            The first strategy is a \blu{brute force approach}:\\ 
            \bigskip
            At every offset, check to see if the characters of the query match
        \end{column}
        \begin{column}{0.5\textwidth}
            \gr{l6_figs/s3_snps.png}
        \end{column}
    \end{columns}
    \bigskip \centering
    Is this a good approach?
\end{frame}

\begin{frame}{Reflecting on the brute force approach}
    Suppose we want to map the read \texttt{GATTACA}:
    \gr{l6_figs/s5_read_mapping.png}
    What do you see that doesn't make sense?
\end{frame}

\begin{frame}{Reflecting on the brute force approach, cont}
    Suppose we want to map the read \texttt{GATTACA}:
    \gr{l6_figs/s5_read_mapping.png}
    What do you see that doesn't make sense?
    \begin{outline}
        \1 \texttt{GATTACA} can't possibly begin at position 4
    \end{outline}
    If we only had one query, maybe this is as good as it gets.\\
    \bigskip
    What if we have lots of queries?
\end{frame}

\section{Suffix arrays}

\begin{frame}{Alternative: suffix arrays}
    We can think of suffix arrays by analogy to searching through a phone book:
    \begin{columns}
        \begin{column}{0.5\textwidth}
            \begin{outline}
                \1 Suppose we need to check many queries
                \1 Why is searching the phone book efficient?
                    \2 If we want to look up \blu{Chlenski} in the phone book, we don't start at page 1 and continue
                    \2 Starting with \blu{C}, lets us skip $25/26 \approx 96\%$ of the book \red{without any loss of accuracy!}
            \end{outline} 
        \end{column}
        \begin{column}{0.5\textwidth}
            \gr{l6_figs/s6_suffix_arrays.png}
        \end{column}
    \end{columns}
\end{frame}

\begin{frame}{Suffix arrays, definition}
    A suffix array is a lexicographically sorted array of all suffixes of a string:
    \gr[0.6]{l6_figs/s7_suffix_array2.png}
\end{frame}

\begin{frame}{Searching the phone book}
    \begin{columns}
        \begin{column}{0.5\textwidth}
            We can do \blu{binary search} over suffix arrays: compare to middle, then search higher/lower; rinse and repeat.\\
            \bigskip
            Searching for \texttt{GATTACA}:\\
            \bigskip
            Low = 1\\
            High = 15
        \end{column}
        \begin{column}{0.5\textwidth}
            \gr{l6_figs/s8_phonebook.png}
        \end{column}
    \end{columns}
\end{frame}

\begin{frame}{Searching the phone book, cont}
    \begin{columns}
        \begin{column}{0.5\textwidth}
            We can do \blu{binary search} over suffix arrays: compare to middle, then search higher/lower; rinse and repeat.\\
            \bigskip
            Searching for \texttt{GATTACA}:\\
            \bigskip
            Mid = (1 + 15)/2 = 8\\
            Middle = suffix[8] = \texttt{CC}
        \end{column}
        \begin{column}{0.5\textwidth}
            \gr{l6_figs/s9_phonebook2.png}
        \end{column}
    \end{columns}
\end{frame}

\begin{frame}{Searching the phone book, cont}
    \begin{columns}
        \begin{column}{0.5\textwidth}
            We can do \blu{binary search} over suffix arrays: compare to middle, then search higher/lower; rinse and repeat.\\
            \bigskip
            Searching for \texttt{GATTACA}:\\
            \bigskip
            \red{Go higher:} Low = Mid + 1 = 9
            High = 15
        \end{column}
        \begin{column}{0.5\textwidth}
            \gr{l6_figs/s11_phonebook3.png}
        \end{column}
    \end{columns}
\end{frame}

\begin{frame}{Searching the phone book, cont}
    \begin{columns}
        \begin{column}{0.5\textwidth}
            We can do \blu{binary search} over suffix arrays: compare to middle, then search higher/lower; rinse and repeat.\\
            \bigskip
            Searching for \texttt{GATTACA}:\\
            \bigskip
            Mid = (9+15)/2 = 12\\
            Middle = suffix[12] = \texttt{TACC}
        \end{column}
        \begin{column}{0.5\textwidth}
            \gr{l6_figs/s12_phonebook4.png}
        \end{column}
    \end{columns}
\end{frame}

\begin{frame}{Searching the phone book, cont}
    \begin{columns}
        \begin{column}{0.5\textwidth}
            We can do \blu{binary search} over suffix arrays: compare to middle, then search higher/lower; rinse and repeat.\\
            \bigskip
            Searching for \texttt{GATTACA}:\\
            \bigskip
            \red{Go lower:} High = Mid - 1 = 11\\
            Low = 9
        \end{column}
        \begin{column}{0.5\textwidth}
            \gr{l6_figs/s12_phonebook4.png}
        \end{column}
    \end{columns}
\end{frame} 

\begin{frame}{Searching the phone book, cont}
    \begin{columns}
        \begin{column}{0.5\textwidth}
            We can do \blu{binary search} over suffix arrays: compare to middle, then search higher/lower; rinse and repeat.\\
            \bigskip
            Searching for \texttt{GATTACA}:\\
            \bigskip
            Mid = (9 + 11)/2 = 10\\
            Middle = suffix[10] = \texttt{GATTACC}
        \end{column}
        \begin{column}{0.5\textwidth}
            \gr{l6_figs/s13_phonebook5.png}
        \end{column}
    \end{columns}
\end{frame} 

\begin{frame}{Searching the phone book, cont}
    \begin{columns}
        \begin{column}{0.5\textwidth}
            We can do \blu{binary search} over suffix arrays: compare to middle, then search higher/lower; rinse and repeat.\\
            \bigskip
            Searching for \texttt{GATTACA}:\\
            \bigskip
            Low = Mid = High = 9\\
            We have converged
        \end{column}
        \begin{column}{0.5\textwidth}
            \gr{l6_figs/s14_phonebook6.png}
        \end{column}
    \end{columns}
\end{frame} 

\begin{frame}{Reflecting on binary search}
    \begin{outline}
        \1[] Binary search:
            \2 Initialize search range to entire list
                \3 Mid = (high + low)/2; middle = suffix[mid]
                \3 If query matches middle: done
                \3 Else if query < middle: pick low range
                \3 Else if query > middle: pick high range
            \2 Repeat until done 
        \1[] Analysis
            \2 More complicated method
            \2 How many times do we repeat?
        \1[] Limitations? 
            \2 HUGE data sizes! Storing multiple copies of redundant information
            \2 How can we store suffix arrays of billions of characters?
    \end{outline}
\end{frame}

\section{Burrows-Wheeler}

\begin{frame}{Intro to Burrows-Wheeler}
    Algorithmic challenge: how do we combine the speed of suffix array with the size of a brute force analysis?\\
    \bigskip
    Reversible permutation of characters:
    \gr[0.8]{l6_figs/s16_bwt.png}
    \tiny A block sorting lossless data compression algorithm. Burrows M, Wheeler DJ (1994) Digital Equipment Corporation. Technical Report 124
\end{frame}

\begin{frame}{Run length encoding}
    \begin{center}
        \tiny REF [614]:
        \begin{verbatim}
            It\_was\_the\_best\_of\_times,\_it\_was\_the\_worst\_of\_times,\_it\_was\_the\_age\_\\
            of\_wisdom,\_it\_was\_the\_age\_of\_foolishness,\_it\_was\_the\_epoch\_of\_belief\\
            ,\_it\_was\_the\_epoch\_of\_incredulity,\_it\_was\_the\_season\_of\_Light,\_it\_wa\\
            s\_the\_season\_of\_Darkness,\_it\_was\_the\_spring\_of\_hope,\_it\_was\_the\_wint\\
            er\_of\_despair,\_we\_had\_everything\_before\_us,\_we\_had\_nothing\_before\_us\\
            ,\_we\_were\_all\_going\_direct\_to\_Heaven,\_we\_were\_all\_going\_direct\_the\_o\\
            ther\_way\_-\_in\_short,\_the\_period\_was\_so\_far\_like\_the\_present\_period,\_\\
            that\_some\_of\_its\_noisiest\_authorities\_insisted\_on\_its\_being\_received\\
            ,\_for\_good\_or\_for\_evil,\_in\_the\_superlative\_degree\_of\_comparison\_only.\$
        \end{verbatim}
    \end{center}
    \bigskip
    \normalsize \red{Definition:}\\
    Replace a “run” of a character X with a single X that is immediately followed by the length of the run
    Ex: \texttt{GAAAAAAAAATTACA} $\to$ \texttt{GA8T2ACA}\\
    (Text containing numbers is slightly more complex encoding)
\end{frame}

\begin{frame}{Run length encoding, cont}
    \begin{center}
        \tiny RLE(REF) [614]:
        \begin{verbatim}
        It\_was\_the\_best\_of\_times,\_it\_was\_the\_worst\_of\_times,\_it\_was\_the\_age\_\\
        of\_wisdom,\_it\_was\_the\_age\_of\_fo2lishnes2,\_it\_was\_the\_epoch\_of\_belief\\
        ,\_it\_was\_the\_epoch\_of\_incredulity,\_it\_was\_the\_season\_of\_Light,\_it\_wa\\
        s\_the\_season\_of\_Darknes2,\_it\_was\_the\_spring\_of\_hope,\_it\_was\_the\_wint\\
        er\_of\_despair,\_we\_had\_everything\_before\_us,\_we\_had\_nothing\_before\_us \\
        ,\_we\_were\_al2\_going\_direct\_to\_Heaven,\_we\_were\_al2\_going\_direct\_the\_o \\
        ther\_way\_-\_in\_short,\_the\_period\_was\_so\_far\_like\_the\_present\_period,\_ \\
        that\_some\_of\_its\_noisiest\_authorities\_insisted\_on\_its\_being\_received \\
        ,\_for\_go2d\_or\_for\_evil,\_in\_the\_superlative\_degre2\_of\_comparison\_only.\$
        \end{verbatim}
    \end{center}
    \bigskip
    On its own, run length encoding does very little for compression!
\end{frame}

\begin{frame}{Run length encoding, cont}
    \tiny
    BWT(REF) [614]:
    \begin{verbatim}
        .dlmssftysesdtrsns\_y\_\_\$\_yfofeeeetggsfefefggeedrofr,llreef-,fs,,,,,,, \\
        ,,nfrsdnnhereghettedndeteegeenstee,ssssst,esssnssffteedttttttttttr,,\\
        ,,eeefehh\_\_p\_\_fpDwwwwwwwwwwweehl\_ew\_\_\_\_\_eoo\_neeeoaaeoo\_\_\_\_sephhrrhvh \\
        hwwegmghhhhhhhkrrwwhhssHrrrvtrribbdbcbvs\_\_thwwpppvmmirdnnib\_\_eoooooo \\
        oooooo\_\_\_\_eennnnnnaai\_\_\_ecc\_\_tttttttttttttttttts\_tsgltsLlvtt\_\_\_hhoor \\
        e\_wrraddwlors\_\_\_\_\_\_\_\_\_r\_\_lteirillre\_ouaanooiioeooooiiihkiiiiiio\_\_iei \\
        tsppioi\_\_\_\_\_\_\_\_\_\_\_\_ggnodsc\_sss\_gfhf\_fffhwh\_nsmo\_\_uee\_sioooaeeeeoo\_ii \\
        cgppeeaoaeooeesseuutetaaaaaaaaaaai\_\_ei\_in\_\_aaie\_eeerei\_hrsssnacciiIi \\
        iiiiiisn\_\_\_\_\_\_\_\_\_\_\_\_\_\_\_oyoui\_\_a\_iiids\_\_aiiaee\_\_\_\_\_\_\_\_\_\_\_\_\_\_\_\_\_\_\_\_\_tlar
    \end{verbatim}
    \bigskip \normalsize
    Burrows-Wheeler Transform puts runs of characters together, so we can compress more
\end{frame}

\begin{frame}{Run length encoding, cont}
    \tiny
    RLE(BWT(REF)) [464]:
    \begin{verbatim}
        .dlms2ftysesdtrsns\_y\_2\$\_yfofe4tg2sfefefg2e2drofr,l2re2f-,fs,9nfrsdn2\\
        hereghet2edndete2ge2nste2,s5t,es3ns2f2te2dt10r,4e3feh2\_2p\_2fpDw11e2h \\
        l\_ew\_5eo2\_ne3oa2eo2\_4seph2r2hvh2w2egmgh7kr2w2h2s2Hr3vtr2ib2dbcbvs\_2t \\
        hw2p3vm2irdn2ib\_2eo12\_4e2n6a2i\_3ec2\_2t18s\_tsgltsLlvt2\_3h2o2re\_wr2ad2 \\
        wlors\_9r\_2lteiril2re\_oua2no2i2oeo4i3hki6o\_2ieitsp2ioi\_12g2nodsc\_s3\_g \\
        fhf\_f3hwh\_nsmo\_2ue2\_sio3ae4o2\_i2cgp2e2aoaeo2e2s2eu2teta11i\_2ei\_in\_2a \\
        2ie\_e3rei\_hrs3nac2i2Ii7sn\_15oyoui\_2a\_i3ds\_2ai2ae2\_21tlar
    \end{verbatim}
    \bigskip \normalsize
    Run-length encoding of BWT is much smaller than run-length encoding of the original sequence. Specifically, we saved 150 bytes with no loss of information!
\end{frame}

\begin{frame}{Burrows-Wheeler algorithm}
\gr{l6_figs/s22_bwt1.png}
\end{frame}

\begin{frame}{Burrows-Wheeler algorithm, cont}
\gr{l6_figs/s23_bwt2.png}
\end{frame}

\begin{frame}{Burrows-Wheeler algorithm, cont}
\gr{l6_figs/s24_bwt3.png}
\end{frame}

\begin{frame}{Burrows-Wheeler algorithm, cont}
\gr{l6_figs/s25_bwt4.png}
\end{frame}

\begin{frame}{Burrows-Wheeler algorithm, cont}
\gr{l6_figs/s26_bwt5.png}
\end{frame}

\begin{frame}{Burrows-Wheeler algorithm, cont}
\gr{l6_figs/s27_bwt6.png}
\end{frame}

\section{Using the BWT}

\begin{frame}{Reversing the BWT}
\gr{l6_figs/s28_rev1.png}
\end{frame}

\begin{frame}{Reversing the BWT, cont}
\gr{l6_figs/s29_rev2.png}
\end{frame}

\begin{frame}{Reversing the BWT, cont}
\gr{l6_figs/s30_rev3.png}
\end{frame}

\begin{frame}{Reversing the BWT, cont}
\gr{l6_figs/s31_rev4.png}
\end{frame}

\begin{frame}{Reversing the BWT, cont}
\gr{l6_figs/s32_rev5.png}
\end{frame}

\begin{frame}{Reversing the BWT, cont}
\gr{l6_figs/s33_rev6.png}
\end{frame}

\begin{frame}{Reversing the BWT, cont}
\gr{l6_figs/s34_rev7.png}
\end{frame}

\begin{frame}{Reversing the BWT, cont}
\gr{l6_figs/s35_rev8.png}
\end{frame}

\begin{frame}{Reversing the BWT, cont}
\gr{l6_figs/s36_rev9.png}
\end{frame}

\begin{frame}{BWA and SAM/BAM files}
\gr{l6_figs/s37_samfile.png}
\end{frame}

\begin{frame}{Variant calling}
\gr{l6_figs/s38_variant_calling.png}
\end{frame}

\begin{frame}{Variant calling, cont}
    \begin{columns}
        \begin{column}{0.6\textwidth}
            SNP calling compares read nucleotides to a reference sequence\\
            \bigskip
            Bayes and SNIPPY are common tools\\
            \bigskip
            Requires sufficient depth to account for sequencing read errors
        \end{column}
        \begin{column}{0.4\textwidth}
            \gr{l6_figs/s39_snippy.png}
        \end{column}
    \end{columns}
\end{frame}

\section{Lab 6}

\begin{frame}{Lab 6}
    \blu{Topic: Burrows-Wheeler Alignments and SNP Calling}\\
    \bigskip
    Work in groups of 2 (new partners!!)\\
    \bigskip
    The lab is to be handed in with the HW at the start of next week
    \begin{outline}
        \1 Labs are to be done in class here
        \1 HWs must be done individually after class
    \end{outline}
\end{frame}


\end{document}