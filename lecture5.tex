\documentclass{beamer}
\usetheme{chlenski}
\usepackage[utf8]{inputenc}
\usepackage{graphicx}
\usepackage{outlines}
\usepackage{hyperref}
\usepackage{babel}
\usepackage[nodisplayskipstretch]{setspace}

\title{Intro to Microbial Genomics}
\subtitle{BIOL BC3308, Lecture 5}
\author{Philippe Chlenski}
\date{February 14, 2023}

\renewcommand{\c}[1]{\begin{center}#1\end{center}}
\newcommand{\blu}[1]{\textcolor{blue}{\textbf{#1}}}
\newcommand{\red}[1]{\textcolor{red}{\textbf{#1}}}
\newcommand{\yel}[1]{\textcolor{yellow}{\textbf{#1}}}
\newcommand{\grn}[1]{\textcolor{dark-green}{\textbf{#1}}}
\newcommand{\prp}[1]{\textcolor{purple}{\textbf{#1}}}
\newcommand{\gr}[2][.95]{\c{\includegraphics[width=#1\textwidth]{#2}}}

% \setbeameroption{show notes}

\begin{document}

\begin{frame}[plain]
\titlepage
\end{frame}

\begin{frame}{Outline}
\tableofcontents
\end{frame}

\section{Background and motivation}

\begin{frame}{Overview}
    \begin{outline}
        \1[] Questions:
            \2 How can we represent relatedness between organisms?
            \2 How to find the best phylogenetic tree given data?
        \1[] Theory:
            \2 Parsimony method
            \2 Distance method
        \1[] Practice:
            \2 Simple distance matrix/tree
            \2 Tree by ST
    \end{outline}
\end{frame}

\begin{frame}{The first tree?}
    \begin{columns}
        \begin{column}{0.7\textwidth}
            ``As buds give rise by growth to fresh buds, and these, if vigorous, branch out and overtop on all sides many a feebler branch, so by generation I believe it has been with the great Tree of Life, which fills with its dead and broken branches the crust of the earth, and covers the surface with its ever-branching and beautiful ramifications.''\\
            \bigskip
            \hfill Charles Darwin, \textit{On the origin of species}
        \end{column}
        \begin{column}{0.3\textwidth}
            \gr{l5_figs/darwin.png}
        \end{column}
    \end{columns}
\end{frame}

\begin{frame}{Intro to trees}
    Life on earth evolved from a common ancestor---so of course biologists looked for a way to represent this!\\
    \bigskip
    \begin{columns}
        \begin{column}{0.5\textwidth}
            We came up with trees as a way to show the relationship of different species, much like a family tree shows the relationship between you and your relatives.
        \end{column}
        \begin{column}{0.5\textwidth}
            \gr{l5_figs/s4_tree_archaea.png}
        \end{column}
    \end{columns}
    \bigskip
    In biology, these are called \blu{phylogenetic trees}.
\end{frame}

\begin{frame}{Intro to trees, cont}
    Trees are not static! For example, Archaea did not ``exist'' before Carl Woese discovered them in 1977.\\
    \bigskip
    \begin{columns}
        \begin{column}{0.5\textwidth}
            \gr{l5_figs/s4_tree_no_archaea.png}
        \end{column}
        \begin{column}{0.5\textwidth}
            \gr{l5_figs/s4_tree_archaea.png}
        \end{column}
    \end{columns}
    \bigskip
    What other technologies may have made this discovery possible?
\end{frame}

\begin{frame}{Intro to trees, cont}
    A tree can be organized according to any property you want\\
    \bigskip
    Early on, trees were constructed on \blu{morphological phylogeny} (do animals share common physical features?)\\
    \bigskip
    Now, we generally look at genetic relatedness (\blu{molecular phylogeny})\\
    \bigskip
    In microbiology, we generally use genome similarity
\end{frame}

% \begin{frame}{Tree thinking}
% % Tree topology
% % Clade
% % Emphasize distance to common ancestor / free rotation
% Thinking in terms of evolutionary trees can be unintuitive, and can require some practice.
% \end{frame}

\begin{frame}{Parts of a tree}
Important components of a tree:
\gr{l5_figs/s6_tree1.png}
\end{frame}

\begin{frame}{Properties of a tree}
A phylogenetic tree is a best guess:
\gr{l5_figs/s7_tree2.png}
We don't really know what happens at internal nodes\\
\bigskip
We infer relationships based on the species we can observe
\end{frame}

\begin{frame}{Tree distances}
Branch length corresponds to ``distance'' between taxa; what does this mean? It depends on how we define it!\\
\bigskip
Common branch ``units'':
\begin{outline}
    \1 Physical traits
    \1 Genetic traits
    \1 Genetic ``distance'' calculated from sequence
    \1 \ldots
\end{outline}
\end{frame}

\begin{frame}{Tree distance practice}
    \begin{columns}
        \begin{column}{0.5\textwidth}
            \gr{l5_figs/s8_viral_tree.png}
            \tiny \url{https://artic.network/how-to-read-a-tree.html}
        \end{column}
        \begin{column}{0.5\textwidth}
            Questions:
            \begin{outline}
                \1 What does the horizontal axis (from left to right) of this tree represent?
                \1 Which of the viruses represented on the tree is the oldest?
                \1 Which virus is most closely related to virus1? Most distantly?
                \1 Which virus is most closely related to virus8?
            \end{outline}
        \end{column}
    \end{columns}
\end{frame}

\begin{frame}{Tree distance practice, cont}
    \begin{columns}
        \begin{column}{0.5\textwidth}
            \gr[0.8]{l5_figs/s9_2trees.png}
        \end{column}
        \begin{column}{0.5\textwidth}
            Questions:
            \begin{outline}
                \1 Which is more closely related to the frog: human or fish?
                \1 Which is more closely related to the fish: frog or human?
            \end{outline}
        \end{column}
    \end{columns}
\end{frame}

\begin{frame}{Tree thinking}
    Thinking in terms of trees can be very unintuitive! Some common errors people make with trees are:
    \begin{enumerate}
        \item Thinking of ``higher'' and ``lower'' species
        \item Imagining a ``main line'' and ``side tracks''
        \item Reading across tips
        \item Conflating relatedness and similarity
        \item Treating siblings as ancestors
        \item Thinking long branches mean no change
        % \item All lineage ages are the same
        % \item Backwards time axes
        % \item More intervening nodes $\neq$ more distantly related
        % \item Change only at nodes
    \end{enumerate}
    For more tree thinking resources, try \blu{\href{https://evolution.berkeley.edu/evolution-101/}{this website}}
\end{frame}

\begin{frame}{Phylogenetics background}
    \blu{Phylo}\red{genetics}:
    \begin{outline}
        \1[] \blu{Phylo ($\textcolor{blue}{\phi\upsilon\lambda\omega\nu}$)}: tribe, genus, species
        \1[] \red{Genetics ($\textcolor{red}{\gamma\iota\gamma\nu\omega\alpha\iota}$)}: coming into being
    \end{outline}
    \bigskip
    \textbf{Definition}: The systematic study of organism relationships based on evolutionary similarities and differences
\end{frame}

\begin{frame}{Phylogenetics terms}
\begin{columns}
    \begin{column}{0.7\textwidth}
        \blu{Unrooted tree}: no ancestor-descendent relationship beyond the observation ``leaves are not ancestors''\\
        \bigskip
        \red{Rooted tree}: the root is the ancestor to the other species in the tree\\
        \bigskip
        \grn{Node}: leaf or branch point\\
        \bigskip
        A tree with $n$ nodes has $n-1$ ways of rooting it
    \end{column}
    \begin{column}{0.3\textwidth}
        \gr{l5_figs/s10_unrooted_tree.png}
        Unrooted tree\\
        \bigskip
        \gr{l5_figs/s10_rooted_tree.png}
        Rooted tree
    \end{column}
\end{columns}
\end{frame}

\begin{frame}{Node counting}
    \begin{columns}
        \begin{column}{0.7\textwidth}
            An \blu{unrooted tree} with $n$ leaves has $2n-2$ nodes\\
            \bigskip
            A \red{rooted tree} with $n$ leaves has $2n-1$ nodes\\
            \bigskip
            Assign the following labels to nodes in a rooted tree:
            \begin{outline}
                \1 Leaf nodes: $1,2,\ldots,n$
                \1 Internal nodes: $n+1, n+2, \ldots, 2n-1$
                \1 Root node: $2n-1$
            \end{outline}
        \end{column}
        \begin{column}{0.3\textwidth}
            \gr{l5_figs/s11_phylogenetic_tree.png}
        \end{column}
    \end{columns}
\end{frame}

\begin{frame}{Tree combinatorics}
    The number of possible rooted and unrooted trees for $n$ species (leaves) increases exponentially with $n$:
    \gr[0.5]{l5_figs/s13_possible_trees.png}
    In fact, tree reconstruction in general is NP-hard! We use faster heuristic algorithms instead.
\end{frame}

\begin{frame}{General approach to phylogeny}
    \begin{outline}
        \1 Use only sequences that evolved from a common ancestor (homologous)!
        \1 Compute alignment
        \1 Construct a phylogeny using 1 of 2 approaches
            \2 Character-based methods
                \3 Input: Sequence alignment
                \3 Output: Phylogenetic tree
                \3 Accurate but computationally very expensive
            \2 Distance-based methods
                \3 Input: Distance matrix containing pairwise statistical estimation of aligned sequences
                \3 Output: Phylogenetic tree
                \3 Fast but less accurate
    \end{outline}
\end{frame}

\section{Character-based algorithms}
\begin{frame}{Character-based algorithms overview}
    Character-based algorithms score and build trees by evaluating the probability that a given tree would yield the sequences of its leaf nodes.\\
    \bigskip
    Two main character-based algorithms:
    \begin{outline}
        \1 Parsimony
        \1 Maximum likelihood
    \end{outline}
\end{frame}

\begin{frame}{Parsimony}
    \blu{General idea:} find the tree that explains the observed sequences with a minimal number of substitutions.\\
    \bigskip
    Two steps:
    \begin{enumerate}
        \item Compute the smallest number of substitutions for a given tree with a parsimony algorithm 
        \item Search for the tree with the minimal number of substitutions
    \end{enumerate}
\end{frame}

\begin{frame}{Parsimony example}
    Consider the following sequences with 105 possible rooted trees:
    \begin{center}
        \texttt{%
            A C T T T\\
            A C A T T\\
            A A C G T\\
            A A T G T\\
            A A T T T\\
        }
    \end{center}
    \bigskip
    Which of the following trees explains the sequences with the least number of substitutions?
\end{frame}

\begin{frame}{Parsimony example, cont}
    \gr{l5_figs/s15_parsimonious.png}
\end{frame}

\begin{frame}{Parsimony example, cont}
    \gr{l5_figs/s16_parsimonious2.png}
\end{frame}

\begin{frame}{Parsimony method thoughts}
    \begin{outline}
        \1[] Requires sequences to be the same length
            \2 First, align sequences, and remove indels
            \2 Then compute parsimony for the resulting sequences
            \2 Alternatively, indels if present can be considered as characters
        \1[] Is the most parsimonious tree the correct tree?
            \2 Not necessarily, but it explains the sequences with least number of substitutions
            \2 We model the probability of having fewer mutations as higher than having many
    \end{outline}
\end{frame}

\begin{frame}{Maximum likelihood trees}
    Most tree-building programs (RaxML, FastTree, et al) use what are known as Maximum Likelihood (ML) tree-building in order to build trees quickly, efficiently and (relatively) accurately for hundreds or thousands of nodes.\\
    \bigskip
    Since we’re focused on practical applications in this class, we won’t go into the gory details of ML algorithms!\\
    \bigskip
    \footnotesize (If you really want to dig into the details, \blu{\href{https://scholarship.claremont.edu/cgi/viewcontent.cgi?article=1047&context=scripps_theses}{this page}} is a good start)
    \end{frame}

\section{Distance-based algorithms}
\begin{frame}{Distance-based algorithms overview}
    \begin{columns}
        \begin{column}{0.6\textwidth}
            Distance-based algorithms operate on the set of pairwise distances $\delta_{i,j}$ between leaves.\\
            \bigskip
            These distances are represented in a \blu{distance matrix}.\\
            \bigskip
            Examples of distance-based algorithms:
            \begin{outline}
                \1 Neighbor-joining
                \1 Unweighted pair group method with arithmetic mean (UPGMA)
            \end{outline}
        \end{column}
        \begin{column}{0.4\textwidth}
            \gr{l5_figs/s18_distance.png}
            Distance $\delta_{i,j}$ states \red{how far apart} species $i$ and $j$ are evolutionarily, e.g. $-1 \times$ alignment score. 
        \end{column}
    \end{columns}
\end{frame}

\begin{frame}{Distance matrix example}
    \begin{columns}
        \begin{column}{0.5\textwidth}
            \gr[0.8]{l5_figs/s19_dist_matrix.png}
        \end{column}
        \begin{column}{0.5\textwidth}
            \red{Goal}: build a tree (with known edge distances) from the distance matrix to the left!\\
            \bigskip
            First, decide whether to build a rooted or unrooted tree. Next, determine how many nodes and edges there are: we have 4 leaves, which means we will have $2 \times 4 - 1 = 7$ nodes for a rooted tree or 6 for an unrooted tree.
        \end{column}
    \end{columns}
\end{frame}

\begin{frame}{Distance matrix example, cont}
    \begin{columns}
        \begin{column}{0.5\textwidth}
            \gr[0.8]{l5_figs/s19_dist_matrix.png}
        \end{column}
        \begin{column}{0.5\textwidth}
            \gr{l5_figs/s20_tree1.png}
            \footnotesize A and B are closest to each other, and C and D are closest to each other, so this is a good guess at an initial tree!
        \end{column}
    \end{columns}
\end{frame}

\begin{frame}{Distance matrix example, cont}
    \begin{columns}
        \begin{column}{0.5\textwidth}
            \gr[0.8]{l5_figs/s19_dist_matrix.png}
        \end{column}
        \begin{column}{0.5\textwidth}
            \gr[0.8]{l5_figs/s21_tree2.png}
            \footnotesize Now, write an equation for each pair of nodes and decompose according to $\delta_{i,j} = \delta_{i,k} + \delta_{j,k}$
        \end{column}
    \end{columns}
\end{frame}

\begin{frame}{Distance matrix example, cont}
    \begin{columns}
        \begin{column}{0.5\textwidth}
            \gr[0.8]{l5_figs/s19_dist_matrix.png}
        \end{column}
        \begin{column}{0.5\textwidth}
            \gr[0.8]{l5_figs/s21_tree2.png}
            \footnotesize We introduce labels $k_1, k_2, k_3$ for ancestral nodes in the tree before setting up our system of equations.
        \end{column}
    \end{columns}
\end{frame}

\begin{frame}{Distance matrix example, cont}
    \begin{columns}
        \begin{column}{0.5\textwidth}
            \footnotesize
            \begin{align*}
                \delta_{A,B} &= \delta_{A,k_1} + \delta_{B,k_1} &= 2\\
                \delta_{A,C} &= \delta_{A,k_3} + \delta_{C,k_3}\\
                &= \delta_{A,k_1} + \delta_{k_1, k_3} + \delta_{C, k_2} + \delta_{k_2, k_3} &= 4\\
                \delta_{A, D} &= \delta_{A, k_3} + \delta_{D, k_3}\\
                &= \delta_{A, k_1} + \delta_{k_1, k_3} + \delta_{D, k_2} + \delta_{k_2, k_3} &= 4\\
                \delta_{B,C} &= \delta_{B,k_3} + \delta_{C,k_3}\\
                &= \delta_{B,k_1} + \delta_{k_1,k_3} + \delta_{C,k_2} + \delta_{k_2,k_3} &=4\\
                \delta_{B,D} &= \delta_{B,k_3} + \delta_{D,k_3}\\
                &= \delta_{B,k_1} + \delta_{k_1,k_3} + \delta_{D,k_2} + \delta_{k_2,k_3} &=4\\
                \delta_{C,D} &= \delta_{C,k_2} + \delta_{D,k_2} &= 2
            \end{align*}
        \end{column}
        \begin{column}{0.5\textwidth}
            \gr[0.8]{l5_figs/s21_tree2.png}
            \footnotesize This system of equations combines our tree topology and distance matrix. We can solve for branch lengths.
        \end{column}
    \end{columns}
\end{frame}

\begin{frame}{Distance matrix example, cont}
    \begin{columns}
        \begin{column}{0.5\textwidth}
            \footnotesize
            \begin{align*}
                \delta_{A,B} &= \textcolor{blue}{\delta_{A,k_1}} + \textcolor{red}{\delta_{B,k_1}} &= 2\\
                \delta_{A,C} &= \delta_{A,k_3} + \delta_{C,k_3}\\
                &= \textcolor{blue}{\delta_{A,k_1}} + \textcolor{purple}{\delta_{k_1, k_3}} + \textcolor{green}{\delta_{C, k_2}} + \textcolor{dark-blue}{\delta_{k_2, k_3}} &= 4\\
                \delta_{A, D} &= \delta_{A, k_3} + \delta_{D, k_3}\\
                &= \textcolor{blue}{\delta_{A, k_1}} + \textcolor{purple}{\delta_{k_1, k_3}} + \textcolor{yellow}{\delta_{D, k_2}} + \textcolor{dark-blue}{\delta_{k_2, k_3}} &= 4\\
                \delta_{B,C} &= \delta_{B,k_3} + \delta_{C,k_3}\\
                &= \textcolor{red}{\delta_{B,k_1}} + \textcolor{purple}{\delta_{k_1,k_3}} + \textcolor{green}{\delta_{C,k_2}} + \textcolor{dark-blue}{\delta_{k_2,k_3}} &=4\\
                \delta_{B,D} &= \delta_{B,k_3} + \delta_{D,k_3}\\
                &= \textcolor{red}{\delta_{B,k_1}} + \textcolor{purple}{\delta_{k_1,k_3}} + \textcolor{yellow}{\delta_{D,k_2}} + \textcolor{dark-blue}{\delta_{k_2,k_3}} &=4\\
                \delta_{C,D} &= \textcolor{green}{\delta_{C,k_2}} + \textcolor{yellow}{\delta_{D,k_2}} &= 2            \end{align*}
        \end{column}
        \begin{column}{0.5\textwidth}
            \gr[0.8]{l5_figs/s23_tree3.png}
            \footnotesize Solving the system of equations on the left yields branch lengths $\textcolor{blue}{\delta_{A,k_1}}, \textcolor{red}{\delta_{B,k_1}}, \textcolor{green}{\delta_{C,k_2}}, \textcolor{yellow}{\delta_{D,k_2}}, \textcolor{purple}{\delta_{k_1,k_3}}, \textcolor{dark-blue}{\delta_{k_2,k_3}}$
        \end{column}
    \end{columns}
\end{frame}

\begin{frame}{Distance matrix example, cont}
    \begin{columns}
        \begin{column}{0.5\textwidth}
            \gr{l5_figs/s19_dist_matrix.png}
        \end{column}
        \begin{column}{0.5\textwidth}
            \gr{l5_figs/s23_tree3.png}
            What are the issues with this method?
        \end{column}
    \end{columns}
\end{frame}

\begin{frame}{Algorithms for additive trees}
    In the previous example, we guessed at the tree structure. This isn’t ideal if we have hundreds or thousands of nodes! There are two main categories of additive tree-building algorithms that automate this process, and we will (very briefly) discuss:
    \begin{outline}
        \1 \blu{UPGMA}: constructs a rooted tree by clustering nodes iteratively
        \1 \blu{Neighbor Joining}: constructs an unrooted tree by clustering similarly to UPGMA
    \end{outline}
    These two algorithms were somewhat basic methods to construct trees; they require a distance matrix and additive data.\\
    \bigskip
    \red{Problem}: We may not have an explicit distance matrix, and we almost certainly don’t have additive data!
\end{frame}


% \section{Other algorithms}
% \begin{frame}{}
%     \ldots
% \end{frame}


\section{Lab 5}
\begin{frame}{Lab 5}
    \blu{Topic: Phylogenetic trees}\\
    \bigskip
    Work in groups of 2\\
    \bigskip
    The lab is to be handed in with the HW at the start of next week
    \begin{outline}
        \1 Labs are to be done in class here
        \1 HWs must be done individually after class
    \end{outline}
\end{frame}


\end{document}