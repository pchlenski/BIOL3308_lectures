\documentclass{article}
\usepackage{graphicx}
\usepackage{outlines}
\usepackage{helvet}
\usepackage{microtype}
\usepackage{hyperref}
\usepackage{tabularx}
\usepackage{fullpage}

% \renewcommand{\familydefault}{\sfdefault}
\title{BIOL BC3308: Intro to Microbial Genomics\\
Exam study guide}
\date{}

\begin{document}

\maketitle{} \noindent
% \textbf{Instructor: } Philippe Chlenski (\texttt{pac@cs.columbia.edu})\\
% \textbf{Office hours: } TBA, CSB 506\\
% \textbf{Location/time: } Wednesdays 1:10--4:00 PM, QLab Sulzberger Hall
% 
\section{Common file formats (fasta, fastq, genbank, BAM, SAM, vcf) (weeks 1 and 6)}
\begin{itemize}
    \item What file types store nucleotide data?
    \item What file types store amino acid data?
    \item What are the requirements for different file formats? 
    \item How are fastq different from fasta files?
    \item What are BAM files and why are they useful?
\end{itemize}

\section{Databases (week 2)}
\begin{itemize}
    \item What are some common databases?
    \item What are the types of databases that exist? 
    \item What types of information are stored in different databases?
    \item How are some ways we access database information?
    \item How does BLAST utilize databases? Why is it useful?
\end{itemize}

\section{Comparative sequence alignment (week 3)}
\begin{itemize}
    \item What is global sequencing alignment? How is it implemented?
    \item What is local sequencing alignment? How is it implemented?
    \item How are local and global alignment approaches different? How is it performed? When might local alignment be preferred over global alignment (and vice versa)?
    \item What is the role of a scoring scheme in sequencing alignment?
\end{itemize}

\section{Genome sequencing and assembly methods (weeks 4 and 6)}
\begin{itemize}
    \item What are the common types of sequencing technologies?
    \item What are the (dis)advantages of different sequencing technologies?
    \item What is the expected output from an Illumina sequencer?
    \item What features of a genome make sequence assembly challenging?
    \item What is paired end sequencing and why is it useful?
    \item What are the steps in an assembly pipeline?
    \item How do we calculate average coverage, and why is it an important metric?
    \item What is the purpose of a De Brujin’s graph? How is it useful? How is it done?
    \item Why is BWT important for mapping reads? How is it done?
    \item What is the difference between de novo assembly and resequencing?
\end{itemize}

\section{Phylogenetics (week 5)}
\begin{itemize}
    \item What are the types of tree-building methods?
    \item How do you interpret phylogenetic trees?
    \item What does a parsimonious tree represent?
    \item How many nodes does a rooted tree with n leaves have (and similar)?
    \item What types of sequences can you use to build a tree?
\end{itemize}

\section{General programs and tools: what programs or functions did we use? What were they used for?}

\end{document}